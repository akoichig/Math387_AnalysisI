\documentclass{article}
\newcommand{\myname}{Your Name Goes Here}
\newcommand{\collaborators}{Last Names}


\newcommand{\semester}{Spring 2016}
\newcommand{\hwnum}{8}
\newcommand{\duedate}{Wednesday, March 30}
\newcommand{\problist}{
{\bf 4.1} \{1,3,6,9\}\\
{\bf 4.2} \{1,2,5,8\}	
}

\newcommand{\unassigned}{
\flushleft 
Unassigned, but suggested: Problems 2,4,7,10 in Section 4.1\\
Unassigned, but suggested: Problems 3,4 in Section 4.2}
% The Course page: http://home.manhattan.edu/~andrew.greene/387/ also suggests some unassigned problems.

%=============================================================================
\usepackage{etoolbox}
\newtoggle{showsolutions}
%== Show solutions? Comment out one of the lines below

\toggletrue{showsolutions}
%\togglefalse{showsolutions}

%== The proofs appear below in between "\begin{solution}" and "\end{solution}."
%=============================================================================


\usepackage[utf8]{inputenc}
\usepackage{amsmath,amssymb,amsthm}
\usepackage{nicefrac}
\usepackage{calc}
\usepackage{enumerate}
\usepackage{graphicx}
\usepackage{verbatim}
\usepackage[draft]{hyperref}
\usepackage{color}
\usepackage[colorinlistoftodos]{todonotes}
\RequirePackage[paper=letterpaper, margin=.75in, headsep=.5in,voffset=0.5in]{geometry}


%=================== Useful Commands
% sets (some)
\newcommand{\C}{{\mathbb{C}}}
\newcommand{\R}{{\mathbb{R}}}
\newcommand{\Z}{{\mathbb{Z}}}
\newcommand{\N}{{\mathbb{N}}}
\newcommand{\Q}{{\mathbb{Q}}}
% reals
\newcommand{\sabs}[1]{\lvert {#1} \rvert}
\newcommand{\snorm}[1]{\lVert {#1} \rVert}
\newcommand{\abs}[1]{\left\lvert {#1} \right\rvert}
\newcommand{\norm}[1]{\left\lVert {#1} \right\rVert}
\newcommand{\myindex}[1]{#1} %ignores myindex command from book


%=================== Header and Footer
\usepackage{fancyhdr}
\pagestyle{fancy}
\fancyhf{}
\lhead{ \textsf{\large\myname}\\
    Math 387\quad Analysis I \\
     Homework \hwnum }
\rhead{H/T: \collaborators\\
    \semester\\
    Due: \duedate}
\chead{ \underline{\textsf{Problem List}}\\[5pt] \problist}
\rfoot{Page \thepage}

%=================== Section/Subsection Formatting
\usepackage[explicit]{titlesec}

\titlespacing\section{0pt}{0 pt}{-30 pt}
\titleformat{\section}
  {}{}{0em}{}%#1 to diplay "Homework #"
%  {\centering \LARGE\bfseries}{}{1em}{}%#1 to diplay "Homework #"

\titleformat{\subsection}
  {\large\bfseries}{\thesubsection}{1em}{Problem {#1}}



%=================== Solution environment
\usepackage{mdframed}
\newenvironment{solution}
    {\begin{mdframed}\begin{proof}[\itshape Solution]}
    {\end{proof}\end{mdframed}}


\begin{document}
\setcounter{section}{\hwnum - 1}
\section{Homework \hwnum}
\emph{ \unassigned}

%--------------------------Question
\subsection{4.1.1}

Prove the product rule.
Hint: Use
$f(x) g(x) - f(c) g(c) = f(x)\bigl( g(x) - g(c) \bigr) + g(c) \bigl( f(x) -
f(c) \bigr)$.


\iftoggle{showsolutions}{\begin{solution}


\end{solution}}{}


%--------------------------Question
\subsection{4.1.3}

For $n \in \Z$,
prove that $x^n$ is differentiable and find the derivative,
unless, of course, $n < 0$ and $x=0$.
Hint: Use the product rule.


\iftoggle{showsolutions}{\begin{solution}


\end{solution}}{}

%--------------------------Question
\subsection{4.1.6}

Assume the inequality $\abs{x-\sin(x)} \leq x^2$.  Prove that sin is
differentiable at $0$, and find the derivative at $0$.


\iftoggle{showsolutions}{\begin{solution}


\end{solution}}{}

%--------------------------Question
\subsection{4.1.9}

Suppose $f \colon \R \to \R$ is a differentiable
Lipschitz continuous function.
Prove that $f'$ is a bounded function.


\iftoggle{showsolutions}{\begin{solution}


\end{solution}}{}


%--------------------------Question
\subsection{4.2.1}

Finish the proof of Proposition 4.2.6: 
Let $f:I\to \R$ be a differentiable function.
\begin{enumerate}[(i)]
\item $f$ is increasing if and only if $f'(x)\ge 0$ for all $x\in I$.
\item $f$ is decreasing if and only if $f'(x)\le 0$ for all $x\in I$.
\end{enumerate}
\emph{Note: Part (i) is proved in the book.}



\iftoggle{showsolutions}{\begin{solution}


\end{solution}}{}

%--------------------------Question
\subsection{4.2.2}

Finish the proof of Proposition 4.2.8: Let $f:(a,b)\to\R$ be continuous. Let $c\in(a,b)$ and suppose $f$ is differentiable on $(a,c)$ and $(c,b)$.  
\begin{enumerate}[(i)]
\item If $f'(x)\le 0$ for $x\in (a,c)$ and $f'(x)\ge 0$ for $x\in(c,b)$, then $f$ has an absolute minimum at $c$. 
\item If $f'(x)\ge 0$ for $x\in (a,c)$ and $f'(x)\le 0$ for $x\in(c,b)$, then $f$ has an absolute maximum at $c$. \end{enumerate}
\emph{Note: Part (i) is proved in the book.}


\iftoggle{showsolutions}{\begin{solution}


\end{solution}}{}


%--------------------------Question
\subsection{4.2.5}

Suppose $f \colon \R \to \R$ is a function such that
$\abs{f(x)-f(y)} \leq \abs{x-y}^2$ for all $x$ and $y$.  Show that
$f(x) = C$ for some constant $C$.  Hint: Show that $f$ is differentiable
at all points and compute the derivative.


\iftoggle{showsolutions}{\begin{solution}


\end{solution}}{}

%--------------------------Question
\subsection{4.2.8}

Suppose $f \colon (a,b) \to \R$ and $g \colon (a,b) \to \R$ are
differentiable functions such that $f'(x) = g'(x)$ for all $x \in (a,b)$,
then show that there exists a constant $C$ such that $f(x) = g(x) + C$.


\iftoggle{showsolutions}{\begin{solution}


\end{solution}}{}



%QUESTION TEMPLATE FOR FUTURE USE
%--------------------------Question
%\subsection{}
%\iftoggle{showsolutions}{\begin{solution} 
%
%\end{solution}}{}



\end{document}