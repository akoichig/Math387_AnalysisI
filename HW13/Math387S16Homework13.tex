\documentclass{article}
\newcommand{\myname}{Your Name Goes Here}
\newcommand{\collaborators}{Last Names}


\newcommand{\semester}{Spring 2016}
\newcommand{\hwnum}{13}
\newcommand{\duedate}{Wednesday, May 4}
\newcommand{\problist}{
{\bf 7.2} \{1,2,7\}}

\newcommand{\unassigned}{
\flushleft 
Unassigned, but suggested: Problems 3,6,8 in Section 7.2
}
% The Course page: http://home.manhattan.edu/~andrew.greene/387/ also suggests some unassigned problems.

%=============================================================================
\usepackage{etoolbox}
\newtoggle{showsolutions}
%== Show solutions? Comment out one of the lines below

\toggletrue{showsolutions}
%\togglefalse{showsolutions}

%== The proofs appear below in between "\begin{solution}" and "\end{solution}."
%=============================================================================


\usepackage[utf8]{inputenc}
\usepackage{amsmath,amssymb,amsthm}
\usepackage{nicefrac}
\usepackage{calc}
\usepackage{enumerate}
\usepackage{graphicx}
\usepackage{verbatim}
\usepackage[draft]{hyperref}
\usepackage{color}
\usepackage[colorinlistoftodos]{todonotes}
\RequirePackage[paper=letterpaper, margin=.75in, headsep=.5in,voffset=0.5in]{geometry}


%=================== Useful Commands
% sets (some)
\newcommand{\C}{{\mathbb{C}}}
\newcommand{\R}{{\mathbb{R}}}
\newcommand{\Z}{{\mathbb{Z}}}
\newcommand{\N}{{\mathbb{N}}}
\newcommand{\Q}{{\mathbb{Q}}}
% reals
\newcommand{\sabs}[1]{\lvert {#1} \rvert}
\newcommand{\snorm}[1]{\lVert {#1} \rVert}
\newcommand{\abs}[1]{\left\lvert {#1} \right\rvert}
\newcommand{\norm}[1]{\left\lVert {#1} \right\rVert}
\newcommand{\myindex}[1]{#1} %ignores myindex command from book
\newcommand{\sR}{{\mathcal{R}}} %Riemann Integrable functions
\newcommand{\rednote}[1]{{\color{red} #1 }}

%=================== Header and Footer
\usepackage{fancyhdr}
\pagestyle{fancy}
\fancyhf{}
\lhead{ \textsf{\large\myname}\\
    Math 387\quad Analysis I \\
     Homework \hwnum }
\rhead{H/T: \collaborators\\
    \semester\\
    Due: \duedate}
\chead{ \underline{\textsf{Problem List}}\\[5pt] \problist}
\rfoot{Page \thepage}

%=================== Section/Subsection Formatting
\usepackage[explicit]{titlesec}

\titlespacing\section{0pt}{0 pt}{-30 pt}
\titleformat{\section}
  {}{}{0em}{}%#1 to diplay "Homework #"
%  {\centering \LARGE\bfseries}{}{1em}{}%#1 to diplay "Homework #"

\titleformat{\subsection}
  {\large\bfseries}{\thesubsection}{1em}{Problem {#1}}



%=================== Solution environment
\usepackage{mdframed}
\newenvironment{solution}
    {\begin{mdframed}\begin{proof}[\itshape Solution]}
    {\end{proof}\end{mdframed}}


\begin{document}
\setcounter{section}{12}
\section{Homework 13}
\emph{ \unassigned}
%--------------------------Question
\subsection{7.2.1}

Prove Proposition 7.2.8 which says the following:\\ 

Let $(X,d)$ be a metric space.
\begin{enumerate}[(i)]
\item \label{topology:closedi} $\emptyset$ and $X$ are closed in $X$.
\item \label{topology:closedii} If $\{ E_\lambda \}_{\lambda \in I}$ is
an arbitrary collection of closed sets, then
\begin{equation*}
\bigcap_{\lambda \in I} E_\lambda
\end{equation*}
is also closed.  That is, intersection of closed sets is closed.
\item \label{topology:closediii} If $E_1, E_2, \ldots, E_k$ are closed then
\begin{equation*}
\bigcup_{j=1}^k E_j
\end{equation*}
is also closed.  That is, finite union of closed sets is closed.
\end{enumerate}


Hint: consider the complements of the
sets and apply Proposition 7.2.6.


\iftoggle{showsolutions}{\begin{solution}


\end{solution}}{}

%--------------------------Question
\subsection{7.2.2}

Finish the proof of Proposition 7.2.9 by
proving that $C(x,\delta)$ is closed.


\iftoggle{showsolutions}{\begin{solution}


\end{solution}}{}

%--------------------------Question
\subsection{7.2.7}

a) Show that $E$ is closed if and only if $\partial E \subset E$.
b) Show that $U$ is open if and only if $\partial U \cap U = \emptyset$.


\iftoggle{showsolutions}{\begin{solution}


\end{solution}}{}








%QUESTION TEMPLATE FOR FUTURE USE
%--------------------------Question
%\subsection{}
%\iftoggle{showsolutions}{\begin{solution} 
%
%\end{solution}}{}



\end{document}