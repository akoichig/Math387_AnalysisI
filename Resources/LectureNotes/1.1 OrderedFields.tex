\documentclass[notitlepage]{report}
\title{Ordered Fields}
\author{Name of Author}

\usepackage{amsmath}
\usepackage{amsfonts}
\usepackage{amssymb}
\usepackage{amsthm}
\usepackage[shortlabels]{enumitem}
\usepackage{fullpage}
\usepackage{nicefrac}

\newcommand{\myindex}[1]{#1}

\theoremstyle{plain}
\newtheorem{thm}{Theorem}[section]
\newtheorem{lemma}[thm]{Lemma}
\newtheorem{prop}[thm]{Proposition}
\newtheorem{cor}[thm]{Corollary}


\theoremstyle{remark}
\newtheorem{remark}[thm]{Remark}

\theoremstyle{definition}
\newtheorem{defn}[thm]{Definition}

\begin{document}
%\maketitle
\noindent {\large \bf 1.1 Basic Properties}

\setcounter{chapter}{1} %\part and \chapter are only available in report and book document classes.
\setcounter{section}{1}


\begin{defn}
An \emph{ordered set} is a set $A$, together with
a relation $<$ such that
%\begin{enumerate}[(i),itemsep=0.5\itemsep,parsep=0.5\parsep,topsep=0.5\topsep,partopsep=0.5\partopsep]
%\begin{enumerate}[(i),nolistsep]
\begin{enumerate}%[(i)]
\item For any $x, y \in A$, exactly one of
$x < y$, $x=y$, or $y < x$ holds.
\item If $x < y$ and $y < z$, then $x < z$.
\end{enumerate}
\end{defn}

\begin{defn}
Let $E \subset A$, where $A$ is an ordered set.
%\begin{enumerate}[(i),itemsep=0.5\itemsep,parsep=0.5\parsep,topsep=0.5\topsep,partopsep=0.5\partopsep]
%\begin{enumerate}[(i),nolistsep]
\begin{enumerate}[(i)]
\item If there exists a $b \in A$ such that $x \leq b$ for all $x \in E$,
then we say $E$ is \emph{\myindex{bounded above}} and $b$
is an \emph{\myindex{upper bound}} of $E$.
\item If there exists a $b \in A$ such that $x \geq b$ for all $x \in E$,
then we say $E$ is \emph{\myindex{bounded below}} and $b$
is a \emph{\myindex{lower bound}} of $E$.
\item If there exists an upper bound $b_0$ of $E$ such that whenever
$b$ is any upper bound for $E$ we have $b_0 \leq b$, then $b_0$
is called the \emph{\myindex{least upper bound}} or
the \emph{\myindex{supremum}}
of $E$.  We write
\begin{equation*}
\sup\, E := b_0  .
\end{equation*}
\item Similarly, if there exists a lower bound $b_0$ of $E$ such that whenever
$b$ is any lower bound for $E$ we have $b_0 \geq b$, then $b_0$
is called the \emph{\myindex{greatest lower bound}} or
the \emph{\myindex{infimum}}
of $E$.  We write
\begin{equation*}
\inf\, E := b_0  .
\end{equation*}
\end{enumerate}
\end{defn}

\begin{defn} \label{defn:lub}
An ordered set $A$ has the \emph{\myindex{least-upper-bound property}} if
every nonempty
subset $E \subset A$ that is bounded above has a least upper bound,
that is $\sup\, E$ exists in $A$.
\end{defn}

\setcounter{thm}{4}

\newpage

\begin{defn}
A set $F$ is called a \emph{\myindex{field}} if it has two operations
defined on it, addition $x+y$ and multiplication $xy$, and if it satisfies
the following axioms.
\begin{enumerate}[({A}1)]
\item If $x \in F$ and $y \in F$, then $x+y \in F$.
\item \emph{(commutativity of addition)}
If $x+y = y+x$ for all $x,y \in F$.
\item \emph{(associativity of addition)}
If $(x+y)+z = x+(y+z)$ for all $x,y,z \in F$.
\item There exists an element $0 \in F$ such that
$0+x = x$ for all $x \in F$.
\item For every element $x\in F$ there exists an element $-x \in F$
such that $x + (-x) = 0$.
\end{enumerate}
\begin{enumerate}[({M}1)]
\item If $x \in F$ and $y \in F$, then $xy \in F$.
\item \emph{(commutativity of multiplication)}
If $xy = yx$ for all $x,y \in F$.
\item \emph{(associativity of multiplication)}
If $(xy)z = x(yz)$ for all $x,y,z \in F$.
\item There exists an element 1 (and $1 \not= 0$) such that
$1x = x$ for all $x \in F$.
\item For every $x\in F$ such that $x \not= 0$ there exists an element
$\nicefrac{1}{x} \in F$
such that $x(\nicefrac{1}{x}) = 1$.
%\end{enumerate}
%\begin{enumerate}
\item[(D)] \emph{(distributive law)} $x(y+z) = xy+xz$
for all $x,y,z \in F$.
\end{enumerate}
\end{defn}

\setcounter{thm}{6}
\begin{defn}
A field $F$ is said to be an \emph{\myindex{ordered field}} if
$F$ is also an ordered set such that:
\begin{enumerate}[(i)]
\item \label{defn:ordfield:i} For $x,y,z \in F$,  $x < y$ implies $x+z <
y+z$.
\item \label{defn:ordfield:ii} For $x,y \in F$, $x > 0$ and $y > 0$
implies $xy > 0$.
\end{enumerate}
If $x > 0$, we say $x$ is \emph{\myindex{positive}}.
If $x < 0$, we say $x$ is \emph{\myindex{negative}}.
We also say $x$ is \emph{\myindex{nonnegative}} if $x \geq 0$,
and $x$ is \emph{\myindex{nonpositive}} if $x \leq 0$.
\end{defn}

\begin{prop} \label{prop:bordfield}
Let $F$ be an ordered field and $x,y,z \in F$.  Then:
\begin{enumerate}[(i)]
\item \label{prop:bordfield:i} If $x > 0$, then $-x < 0$ (and vice-versa).
\item \label{prop:bordfield:ii} If $x > 0$ and $y < z$, then $xy < xz$.
\item \label{prop:bordfield:iii} If $x < 0$ and $y < z$, then $xy > xz$.
\item \label{prop:bordfield:iv} If $x \not= 0$, then $x^2 > 0$.
\item \label{prop:bordfield:v} If $0 < x < y$, then $0 < \nicefrac{1}{y} <
\nicefrac{1}{x}$.
\end{enumerate}
\end{prop}

\begin{prop}
Let $x,y \in F$ where $F$ is an ordered field.  Suppose 
$xy > 0$.  Then either both $x$ and $y$ are positive, or both are negative.
\end{prop}



\end{document}
