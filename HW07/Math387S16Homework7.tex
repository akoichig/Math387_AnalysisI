\documentclass{article}
\newcommand{\myname}{Your Name Goes Here}
\newcommand{\collaborators}{Last Names}


\newcommand{\semester}{Spring 2016}
\newcommand{\hwnum}{7}
\newcommand{\duedate}{Wednesday, March 23}
\newcommand{\problist}{
{\bf 3.3} \{1,5,7\}	\\
{\bf 3.4} \{2,3,7\}	
}

\newcommand{\unassigned}{
\flushleft 
Unassigned, but suggested: Problems 3,6,8 in Section 3.3\\
Unassigned, but suggested: Problems 1,4,5,6 in Section 3.4
}
% The Course page: http://home.manhattan.edu/~andrew.greene/387/ also suggests some unassigned problems.

%=============================================================================
\usepackage{etoolbox}
\newtoggle{showsolutions}
%== Show solutions? Comment out one of the lines below

\toggletrue{showsolutions}
%\togglefalse{showsolutions}

%== The proofs appear below in between "\begin{solution}" and "\end{solution}."
%=============================================================================


\usepackage[utf8]{inputenc}
\usepackage{amsmath,amssymb,amsthm}
\usepackage{nicefrac}
\usepackage{calc}
\usepackage{enumerate}
\usepackage{graphicx}
\usepackage{verbatim}
\usepackage[draft]{hyperref}
\usepackage{color}
\usepackage[colorinlistoftodos]{todonotes}
\RequirePackage[paper=letterpaper, margin=.75in, headsep=.5in,voffset=0.5in]{geometry}


%=================== Useful Commands
% sets (some)
\newcommand{\C}{{\mathbb{C}}}
\newcommand{\R}{{\mathbb{R}}}
\newcommand{\Z}{{\mathbb{Z}}}
\newcommand{\N}{{\mathbb{N}}}
\newcommand{\Q}{{\mathbb{Q}}}
% reals
\newcommand{\sabs}[1]{\lvert {#1} \rvert}
\newcommand{\snorm}[1]{\lVert {#1} \rVert}
\newcommand{\abs}[1]{\left\lvert {#1} \right\rvert}
\newcommand{\norm}[1]{\left\lVert {#1} \right\rVert}
\newcommand{\myindex}[1]{#1} %ignores myindex command from book


%=================== Header and Footer
\usepackage{fancyhdr}
\pagestyle{fancy}
\fancyhf{}
\lhead{ \textsf{\large\myname}\\
    Math 387\quad Analysis I \\
     Homework \hwnum }
\rhead{H/T: \collaborators\\
    \semester\\
    Due: \duedate}
\chead{ \underline{\textsf{Problem List}}\\[5pt] \problist}
\rfoot{Page \thepage}

%=================== Section/Subsection Formatting
\usepackage[explicit]{titlesec}

\titlespacing\section{0pt}{0 pt}{-30 pt}
\titleformat{\section}
  {}{}{0em}{}%#1 to diplay "Homework #"
%  {\centering \LARGE\bfseries}{}{1em}{}%#1 to diplay "Homework #"

\titleformat{\subsection}
  {\large\bfseries}{\thesubsection}{1em}{Problem {#1}}



%=================== Solution environment
\usepackage{mdframed}
\newenvironment{solution}
    {\begin{mdframed}\begin{proof}[\itshape Solution]}
    {\end{proof}\end{mdframed}}


\begin{document}
\setcounter{section}{\hwnum - 1}
\section{Homework \hwnum}
\emph{ \unassigned}


%--------------------------Question
\subsection{3.3.1}

Find an example of a discontinuous function $f \colon [0,1] \to \R$
where the intermediate value theorem fails.


\iftoggle{showsolutions}{\begin{solution}


\end{solution}}{}


%--------------------------Question
\subsection{3.3.5}

Suppose $g(x)$ is a polynomial of odd degree $d$ such that
\begin{equation*}
g(x) = x^d + b_{d-1} x^{d-1} + \cdots + b_1 x + b_0 ,
\end{equation*}
for some real numbers $b_{0}, b_1, \ldots, b_{d-1}$.  Show that there exists
a $K \in \N$ such that $g(-K) < 0$.  Hint: Make sure to use the fact that
$d$ is odd.  You will have to use that ${(-n)}^d = -(n^d)$.


\iftoggle{showsolutions}{\begin{solution}


\end{solution}}{}

%--------------------------Question
\subsection{3.3.7}

Suppose $f \colon [a,b] \to \R$ is a continuous function.  Prove
that the direct image $f([a,b])$ is a closed and bounded interval or
a single number.


\iftoggle{showsolutions}{\begin{solution}


\end{solution}}{}


%--------------------------Question
\subsection{3.4.2}

Let $f \colon (a,b) \to \R$ be a uniformly continuous function.
Finish the proof of Theorem 3.4.6 by showing that
the limit
$\displaystyle \lim_{x \to b} f(x)$
exists.


\iftoggle{showsolutions}{\begin{solution}


\end{solution}}{}

%--------------------------Question
\subsection{3.4.3}

Show that $f \colon (c,\infty) \to \R$ for some $c > 0$
and defined by $f(x) := \nicefrac{1}{x}$ is Lipschitz continuous.


\iftoggle{showsolutions}{\begin{solution}


\end{solution}}{}

%--------------------------Question
\subsection{3.4.7}

Let $f \colon (0,1) \to \R$ be a bounded continuous function.  Show that
the function
$g(x) := x(1-x)f(x)$ is uniformly continuous.


\iftoggle{showsolutions}{\begin{solution}


\end{solution}}{}




%QUESTION TEMPLATE FOR FUTURE USE
%--------------------------Question
%\subsection{}
%\iftoggle{showsolutions}{\begin{solution} 
%
%\end{solution}}{}



\end{document}