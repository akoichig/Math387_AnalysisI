\documentclass{article}
\newcommand{\myname}{Your Name Goes Here}
\newcommand{\collaborators}{Last Names}


\newcommand{\semester}{Spring 2016}
\newcommand{\hwnum}{14}
\newcommand{\duedate}{Wednesday, May 11}
\newcommand{\problist}{
{\bf 7.3} \{1,2,3\}
{\bf 7.4} \{1,2\}}

\newcommand{\unassigned}{
\flushleft 
Unassigned, but suggested: Problem 4 in Section 7.3\\
Unassigned, but suggested: Problem 4 in Section 7.4
}
% The Course page: http://home.manhattan.edu/~andrew.greene/387/ also suggests some unassigned problems.

%=============================================================================
\usepackage{etoolbox}
\newtoggle{showsolutions}
%== Show solutions? Comment out one of the lines below

\toggletrue{showsolutions}
%\togglefalse{showsolutions}

%== The proofs appear below in between "\begin{solution}" and "\end{solution}."
%=============================================================================


\usepackage[utf8]{inputenc}
\usepackage{amsmath,amssymb,amsthm}
\usepackage{nicefrac}
\usepackage{calc}
\usepackage{enumerate}
\usepackage{graphicx}
\usepackage{verbatim}
\usepackage[draft]{hyperref}
\usepackage{color}
\usepackage[colorinlistoftodos]{todonotes}
\RequirePackage[paper=letterpaper, margin=.75in, headsep=.5in,voffset=0.5in]{geometry}


%=================== Useful Commands
% sets (some)
\newcommand{\C}{{\mathbb{C}}}
\newcommand{\R}{{\mathbb{R}}}
\newcommand{\Z}{{\mathbb{Z}}}
\newcommand{\N}{{\mathbb{N}}}
\newcommand{\Q}{{\mathbb{Q}}}
% reals
\newcommand{\sabs}[1]{\lvert {#1} \rvert}
\newcommand{\snorm}[1]{\lVert {#1} \rVert}
\newcommand{\abs}[1]{\left\lvert {#1} \right\rvert}
\newcommand{\norm}[1]{\left\lVert {#1} \right\rVert}
\newcommand{\myindex}[1]{#1} %ignores myindex command from book
\newcommand{\sR}{{\mathcal{R}}} %Riemann Integrable functions
\newcommand{\rednote}[1]{{\color{red} #1 }}

%=================== Header and Footer
\usepackage{fancyhdr}
\pagestyle{fancy}
\fancyhf{}
\lhead{ \textsf{\large\myname}\\
    Math 387\quad Analysis I \\
     Homework \hwnum }
\rhead{H/T: \collaborators\\
    \semester\\
    Due: \duedate}
\chead{ \underline{\textsf{Problem List}}\\[5pt] \problist}
\rfoot{Page \thepage}

%=================== Section/Subsection Formatting
\usepackage[explicit]{titlesec}

\titlespacing\section{0pt}{0 pt}{-30 pt}
\titleformat{\section}
  {}{}{0em}{}%#1 to diplay "Homework #"
%  {\centering \LARGE\bfseries}{}{1em}{}%#1 to diplay "Homework #"

\titleformat{\subsection}
  {\large\bfseries}{\thesubsection}{1em}{Problem {#1}}



%=================== Solution environment
\usepackage{mdframed}
\newenvironment{solution}
    {\begin{mdframed}\begin{proof}[\itshape Solution]}
    {\end{proof}\end{mdframed}}


\begin{document}
\setcounter{section}{13}
\section{Homework 14}
\emph{ \unassigned}

%--------------------------Question
\subsection{7.3.1}

Let $(X,d)$ be a metric space and
let $A \subset X$.  Let $E$ be the set of all $x \in X$ such that there
exists a sequence $\{ x_n \}$ in $A$ that converges to $x$.  Show 
$E = \overline{A}$.


\iftoggle{showsolutions}{\begin{solution}


\end{solution}}{}

%--------------------------Question
\subsection{7.3.2}

a) Show that $d(x,y) := \min \{ 1, \abs{x-y} \}$ defines a metric on $\R$.
b) Show that a sequence converges in $(\R,d)$ if and only if it converges
in the standard metric.  c) Find a bounded sequence in $(\R,d)$ that
contains no convergent subsequence.


\iftoggle{showsolutions}{\begin{solution}


\end{solution}}{}

%--------------------------Question
\subsection{7.3.3}

Prove Proposition 7.3.4 which says a convergent sequence in a metric space is bounded.


\iftoggle{showsolutions}{\begin{solution}


\end{solution}}{}

%--------------------------Question
\subsection{7.4.1}

Let $(X,d)$ be a metric space and $A$ a finite subset of $X$.
Show that $A$ is compact.


\iftoggle{showsolutions}{\begin{solution}


\end{solution}}{}

%--------------------------Question
\subsection{7.4.2}

Let $A = \{ \nicefrac{1}{n} : n \in \N \} \subset \R$.  a)~Show that $A$ is
not compact directly using the definition.  b)~Show that $A \cup \{ 0 \}$ is
compact directly using the definition.


\iftoggle{showsolutions}{\begin{solution}


\end{solution}}{}




%QUESTION TEMPLATE FOR FUTURE USE
%--------------------------Question
%\subsection{}
%\iftoggle{showsolutions}{\begin{solution} 
%
%\end{solution}}{}



\end{document}