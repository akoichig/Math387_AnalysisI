\documentclass{article}
\newcommand{\myname}{Your Name Goes Here}
\newcommand{\collaborators}{Last Names}


\newcommand{\semester}{Spring 2016}
\newcommand{\hwnum}{11}
\newcommand{\duedate}{Wednesday, April 20}
\newcommand{\problist}{
{\bf 5.3} \{1,2,4,5,6\}}

\newcommand{\unassigned}{
\flushleft 
Unassigned, but suggested: Problems 7,10 in Section 5.3
}
% The Course page: http://home.manhattan.edu/~andrew.greene/387/ also suggests some unassigned problems.

%=============================================================================
\usepackage{etoolbox}
\newtoggle{showsolutions}
%== Show solutions? Comment out one of the lines below

\toggletrue{showsolutions}
%\togglefalse{showsolutions}

%== The proofs appear below in between "\begin{solution}" and "\end{solution}."
%=============================================================================


\usepackage[utf8]{inputenc}
\usepackage{amsmath,amssymb,amsthm}
\usepackage{nicefrac}
\usepackage{calc}
\usepackage{enumerate}
\usepackage{graphicx}
\usepackage{verbatim}
\usepackage[draft]{hyperref}
\usepackage{color}
\usepackage[colorinlistoftodos]{todonotes}
\RequirePackage[paper=letterpaper, margin=.75in, headsep=.5in,voffset=0.5in]{geometry}


%=================== Useful Commands
% sets (some)
\newcommand{\C}{{\mathbb{C}}}
\newcommand{\R}{{\mathbb{R}}}
\newcommand{\Z}{{\mathbb{Z}}}
\newcommand{\N}{{\mathbb{N}}}
\newcommand{\Q}{{\mathbb{Q}}}
% reals
\newcommand{\sabs}[1]{\lvert {#1} \rvert}
\newcommand{\snorm}[1]{\lVert {#1} \rVert}
\newcommand{\abs}[1]{\left\lvert {#1} \right\rvert}
\newcommand{\norm}[1]{\left\lVert {#1} \right\rVert}
\newcommand{\myindex}[1]{#1} %ignores myindex command from book
\newcommand{\sR}{{\mathcal{R}}} %Riemann Integrable functions
\newcommand{\rednote}[1]{{\color{red} #1 }}

%=================== Header and Footer
\usepackage{fancyhdr}
\pagestyle{fancy}
\fancyhf{}
\lhead{ \textsf{\large\myname}\\
    Math 387\quad Analysis I \\
     Homework \hwnum }
\rhead{H/T: \collaborators\\
    \semester\\
    Due: \duedate}
\chead{ \underline{\textsf{Problem List}}\\[5pt] \problist}
\rfoot{Page \thepage}

%=================== Section/Subsection Formatting
\usepackage[explicit]{titlesec}

\titlespacing\section{0pt}{0 pt}{-30 pt}
\titleformat{\section}
  {}{}{0em}{}%#1 to diplay "Homework #"
%  {\centering \LARGE\bfseries}{}{1em}{}%#1 to diplay "Homework #"

\titleformat{\subsection}
  {\large\bfseries}{\thesubsection}{1em}{Problem {#1}}



%=================== Solution environment
\usepackage{mdframed}
\newenvironment{solution}
    {\begin{mdframed}\begin{proof}[\itshape Solution]}
    {\end{proof}\end{mdframed}}


\begin{document}
\setcounter{section}{10}
\section{Homework 11}
\emph{ \unassigned}

%--------------------------Question
\subsection{5.3.1}

Compute
$\displaystyle
\frac{d}{dx} \biggl( \int_{-x}^x e^{s^2}~ds \biggr)$.


\iftoggle{showsolutions}{\begin{solution}


\end{solution}}{}

%--------------------------Question
\subsection{5.3.2}

Compute
$\displaystyle
\frac{d}{dx} \biggl( \int_{0}^{x^2} \sin(s^2)~ds \biggr)$.


\iftoggle{showsolutions}{\begin{solution}


\end{solution}}{}

%--------------------------Question
\subsection{5.3.4}

Let $f \colon [a,b] \to \R$ be a continuous function.  Let $c \in [a,b]$
be arbitrary.  Define
\begin{equation*}
F(x) := \int_c^x f .
\end{equation*}
Prove that $F$ is differentiable and that $F'(x) = f(x)$ for all $x \in
[a,b]$.


\iftoggle{showsolutions}{\begin{solution}


\end{solution}}{}

%--------------------------Question
\subsection{5.3.5}

Prove \emph{\myindex{integration by parts}}.  That is, suppose $F$ and
$G$ are continuously differentiable functions on $[a,b]$.  Then prove
\begin{equation*}
\int_a^b F(x)G'(x)~dx
=
F(b)G(b)-F(a)G(a)
-
\int_a^b F'(x)G(x)~dx .
\end{equation*}


\iftoggle{showsolutions}{\begin{solution}


\end{solution}}{}

%--------------------------Question
\subsection{5.3.6}

Suppose $F$ and $G$ are continuously
differentiable
functions defined on $[a,b]$
such that $F'(x) = G'(x)$ for all $x \in [a,b]$.
Using the fundamental theorem of calculus,
show that $F$ and $G$ differ by a constant.  That is, show that
there exists a $C \in \R$ such that
$F(x)-G(x) = C$.


\iftoggle{showsolutions}{\begin{solution}


\end{solution}}{}






%QUESTION TEMPLATE FOR FUTURE USE
%--------------------------Question
%\subsection{}
%\iftoggle{showsolutions}{\begin{solution} 
%
%\end{solution}}{}



\end{document}