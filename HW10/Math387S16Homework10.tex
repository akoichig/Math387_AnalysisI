\documentclass{article}
\newcommand{\myname}{Your Name Goes Here}
\newcommand{\collaborators}{Last Names}


\newcommand{\semester}{Spring 2016}
\newcommand{\hwnum}{10}
\newcommand{\duedate}{Wednesday, April 13}
\newcommand{\problist}{
{\bf 5.2} \{1,2,4,5,7\}}

\newcommand{\unassigned}{
\flushleft 
Unassigned, but suggested: Problem 6 in Section 5.2
}
% The Course page: http://home.manhattan.edu/~andrew.greene/387/ also suggests some unassigned problems.

%=============================================================================
\usepackage{etoolbox}
\newtoggle{showsolutions}
%== Show solutions? Comment out one of the lines below

\toggletrue{showsolutions}
%\togglefalse{showsolutions}

%== The proofs appear below in between "\begin{solution}" and "\end{solution}."
%=============================================================================


\usepackage[utf8]{inputenc}
\usepackage{amsmath,amssymb,amsthm}
\usepackage{nicefrac}
\usepackage{calc}
\usepackage{enumerate}
\usepackage{graphicx}
\usepackage{verbatim}
\usepackage[draft]{hyperref}
\usepackage{color}
\usepackage[colorinlistoftodos]{todonotes}
\RequirePackage[paper=letterpaper, margin=.75in, headsep=.5in,voffset=0.5in]{geometry}


%=================== Useful Commands
% sets (some)
\newcommand{\C}{{\mathbb{C}}}
\newcommand{\R}{{\mathbb{R}}}
\newcommand{\Z}{{\mathbb{Z}}}
\newcommand{\N}{{\mathbb{N}}}
\newcommand{\Q}{{\mathbb{Q}}}
% reals
\newcommand{\sabs}[1]{\lvert {#1} \rvert}
\newcommand{\snorm}[1]{\lVert {#1} \rVert}
\newcommand{\abs}[1]{\left\lvert {#1} \right\rvert}
\newcommand{\norm}[1]{\left\lVert {#1} \right\rVert}
\newcommand{\myindex}[1]{#1} %ignores myindex command from book
\newcommand{\sR}{{\mathcal{R}}} %Riemann Integrable functions
\newcommand{\rednote}[1]{{\color{red} #1 }}

%=================== Header and Footer
\usepackage{fancyhdr}
\pagestyle{fancy}
\fancyhf{}
\lhead{ \textsf{\large\myname}\\
    Math 387\quad Analysis I \\
     Homework \hwnum }
\rhead{H/T: \collaborators\\
    \semester\\
    Due: \duedate}
\chead{ \underline{\textsf{Problem List}}\\[5pt] \problist}
\rfoot{Page \thepage}

%=================== Section/Subsection Formatting
\usepackage[explicit]{titlesec}

\titlespacing\section{0pt}{0 pt}{-30 pt}
\titleformat{\section}
  {}{}{0em}{}%#1 to diplay "Homework #"
%  {\centering \LARGE\bfseries}{}{1em}{}%#1 to diplay "Homework #"

\titleformat{\subsection}
  {\large\bfseries}{\thesubsection}{1em}{Problem {#1}}



%=================== Solution environment
\usepackage{mdframed}
\newenvironment{solution}
    {\begin{mdframed}\begin{proof}[\itshape Solution]}
    {\end{proof}\end{mdframed}}


\begin{document}
\setcounter{section}{9}
\section{Homework 10}
\emph{ \unassigned}


%--------------------------Question
\subsection{5.2.1}

Let $f$ be in $\sR[a,b]$.  Prove that
$-f$ is in $\sR[a,b]$ and 
\begin{equation*}
\int_a^b - f(x) ~dx = - \int_a^b f(x) ~dx .
\end{equation*}


\iftoggle{showsolutions}{\begin{solution}


\end{solution}}{}

%--------------------------Question
\subsection{5.2.2}

Let $f$ and $g$ be in $\sR[a,b]$.
Prove that $f+g$ is in $\sR[a,b]$ and
\begin{equation*}
\int_a^b f(x)+g(x) ~dx = 
\int_a^b f(x) ~dx 
+
\int_a^b g(x) ~dx .
\end{equation*}
Hint: Use Proposition 5.1.7 to find a single partition $P$
such that $U(P,f)-L(P,f) < \nicefrac{\epsilon}{2}$ and
$U(P,g)-L(P,g) < \nicefrac{\epsilon}{2}$.


\iftoggle{showsolutions}{\begin{solution}


\end{solution}}{}

%--------------------------Question
\subsection{5.2.4}

Prove the \emph{\myindex{mean value theorem for integrals}}.  That is,
prove that if $f \colon [a,b] \to \R$ is continuous, then there exists
a $c \in [a,b]$ such that $\int_a^b f = f(c)(b-a)$.


\iftoggle{showsolutions}{\begin{solution}


\end{solution}}{}

%--------------------------Question
\subsection{5.2.5}

If $f \colon [a,b] \to \R$ is a continuous function such that $f(x) \geq 0$
for all $x \in [a,b]$ and $\int_a^b f = 0$.  Prove that $f(x) = 0$
for all $x$.


\iftoggle{showsolutions}{\begin{solution}


\end{solution}}{}

%--------------------------Question
\subsection{5.2.7}

If $f \colon [a,b] \to \R$ and $g \colon [a,b] \to \R$
are continuous functions such that $\int_a^b f = \int_a^b g$.
Then show that there exists a $c \in [a,b]$ such that $f(c) = g(c)$.


\iftoggle{showsolutions}{\begin{solution}


\end{solution}}{}





%QUESTION TEMPLATE FOR FUTURE USE
%--------------------------Question
%\subsection{}
%\iftoggle{showsolutions}{\begin{solution} 
%
%\end{solution}}{}



\end{document}