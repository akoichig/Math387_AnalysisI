\documentclass{article}
\newcommand{\myname}{Your Name Goes Here}
\newcommand{\collaborators}{Last Names}


\newcommand{\hwnum}{4}
\newcommand{\duedate}{Thursday, February 16}
\newcommand{\problist}{
{\bf 2.2} \{3,6,9\} \\
{\bf 2.3} \{1,3,5,9\}
}

\newcommand{\unassigned}{
\flushleft Unassigned, but suggested: Problems 7, 8 in Section 2.2\\
Unassigned, but suggested: Problems 2, 4 in Section 2.3 }
% The Course page: http://turing.manhattan.edu/~andrew.greene/387/ also suggests some unassigned problems.

%=================== Important preamble file
% contains macro definitions of mysolution, myreview, myupdatedsolution
\usepackage{preamble387S17}

%=================== Useful Commands
% sets (some)
\newcommand{\C}{{\mathbb{C}}}
\newcommand{\R}{{\mathbb{R}}}
\newcommand{\Z}{{\mathbb{Z}}}
\newcommand{\N}{{\mathbb{N}}}
\newcommand{\Q}{{\mathbb{Q}}}
% reals
\newcommand{\sabs}[1]{\lvert {#1} \rvert}
\newcommand{\snorm}[1]{\lVert {#1} \rVert}
\newcommand{\abs}[1]{\left\lvert {#1} \right\rvert}
\newcommand{\norm}[1]{\left\lVert {#1} \right\rVert}
\newcommand{\myindex}[1]{#1} %ignores myindex command from book
\newcommand{\rednote}[1]{{\color{red} #1}}


\begin{document}
\setcounter{section}{\hwnum - 1}
\section{Homework \hwnum}
\emph{ \unassigned}


%--------------------------Question
\subsection{2.2.3}Prove that if $\{ x_n \}$ is a convergent sequence, $k \in \N$, then
\begin{equation*}
\lim_{n\to\infty} x_n^k = 
{\left( \lim_{n\to\infty} x_n \right)}^k .
\end{equation*}
Hint: Use {induction}.

\mysolution{
% YOUR SOLUTION GOES BELOW

} % END OF SOLUTION


%--------------------------Question
\subsection{2.2.6}Let $x_n := \frac{1}{n^2}$ and $y_n := \frac{1}{n}$.  Define
$z_n := \frac{x_n}{y_n}$ and 
$w_n := \frac{y_n}{x_n}$.  Do $\{ z_n \}$ and $\{ w_n \}$
converge?  What are the limits?  Can you apply Proposition 2.2.5?
Why or why not?

\mysolution{
% YOUR SOLUTION GOES BELOW

} % END OF SOLUTION


%--------------------------Question
\subsection{2.2.9}Suppose $\{ x_n \}$ is a sequence and suppose for
some $x \in \R$, the limit
\begin{equation*}
L := \lim_{n \to \infty} \frac{\abs{x_{n+1}-x}}{\abs{x_n-x}}
\end{equation*}
exists and $L < 1$.  Show that $\{ x_n \}$ converges to $x$.

\mysolution{
% YOUR SOLUTION GOES BELOW

} % END OF SOLUTION



%--------------------------Question
\subsection{2.3.1}Suppose $\{ x_n \}$ is a bounded sequence.  Define $a_n$ and
$b_n$ as in Definition 2.3.1.  Show that $\{ a_n \}$
and $\{ b_n \}$ are bounded.

\mysolution{
% YOUR SOLUTION GOES BELOW

} % END OF SOLUTION



%--------------------------Question
\subsection{2.3.3}Finish the proof of Proposition 2.3.6.  That is,
suppose $\{ x_n \}$ is a bounded sequence and
$\{ x_{n_k} \}$ is a subsequence.  Prove
$\displaystyle \liminf_{n\to\infty}\, x_n \leq
\liminf_{k\to\infty}\, x_{n_k}$.

\mysolution{
% YOUR SOLUTION GOES BELOW

} % END OF SOLUTION


%--------------------------Question
\subsection{2.3.5}\begin{enumerate}[a)]
\item
Let $x_n := \dfrac{{(-1)}^n}{n}$, find $\limsup \, x_n$ and $\liminf \, x_n$.
\item
Let $x_n := \dfrac{(n-1){(-1)}^n}{n}$, find $\limsup \, x_n$ and $\liminf \, x_n$.
\end{enumerate}

\mysolution{
% YOUR SOLUTION GOES BELOW

} % END OF SOLUTION


%--------------------------Question
\subsection{2.3.9}If $S \subset \R$ is a set, then $x \in \R$ is a \emph{\myindex{cluster
point}}
if for every $\epsilon > 0$, the set $(x-\epsilon,x+\epsilon) \cap S
\setminus \{ x \}$ is not empty.  That is, if there are points of $S$
arbitrarily close to $x$.
For example, $S := \{ \nicefrac{1}{n} : n \in \N \}$ has a unique (only
one) cluster point $0$, but $0 \notin S$.
Prove the following version of the Bolzano-Weierstrass theorem:

\medskip

\noindent
\emph{\textbf{Theorem.} Let $S \subset \R$ be a bounded infinite set,
then there exists at least one cluster point of $S$}.

\medskip

Hint: If $S$ is infinite, then $S$ contains a countably infinite subset.
That is, there is a sequence $\{ x_n \}$ of distinct numbers in $S$.

\mysolution{
% YOUR SOLUTION GOES BELOW

} % END OF SOLUTION



\end{document}