\documentclass{article}
\newcommand{\myname}{Your Name Goes Here}
\newcommand{\collaborators}{Last Names}


\newcommand{\semester}{Spring 2016}
\newcommand{\hwnum}{5}
\newcommand{\duedate}{Wednesday, March 2}
\newcommand{\problist}{
{\bf 2.4} \{1,2,4\}\\
{\bf 2.5} \{1,2\}
}

\newcommand{\unassigned}{
\flushleft Unassigned, but suggested: Problems 3,5 in Section 2.4\\
Unassigned, but suggested: Problems 4,5,6,11 in Section 2.5

}
% The Course page: http://home.manhattan.edu/~andrew.greene/387/ also suggests some unassigned problems.

%=============================================================================
\usepackage{etoolbox}
\newtoggle{showsolutions}
%== Show solutions? Comment out one of the lines below

\toggletrue{showsolutions}
%\togglefalse{showsolutions}

%== The proofs appear below in between "\begin{solution}" and "\end{solution}."
%=============================================================================


\usepackage[utf8]{inputenc}
\usepackage{amsmath,amssymb,amsthm}
\usepackage{nicefrac}
\usepackage{calc}
\usepackage{enumerate}
\usepackage{graphicx}
\usepackage{verbatim}
\usepackage[draft]{hyperref}
\usepackage{color}
\usepackage[colorinlistoftodos]{todonotes}
\RequirePackage[paper=letterpaper, margin=.75in, headsep=.5in,voffset=0.5in]{geometry}


%=================== Useful Commands
% sets (some)
\newcommand{\C}{{\mathbb{C}}}
\newcommand{\R}{{\mathbb{R}}}
\newcommand{\Z}{{\mathbb{Z}}}
\newcommand{\N}{{\mathbb{N}}}
\newcommand{\Q}{{\mathbb{Q}}}
% reals
\newcommand{\sabs}[1]{\lvert {#1} \rvert}
\newcommand{\snorm}[1]{\lVert {#1} \rVert}
\newcommand{\abs}[1]{\left\lvert {#1} \right\rvert}
\newcommand{\norm}[1]{\left\lVert {#1} \right\rVert}
\newcommand{\myindex}[1]{#1} %ignores myindex command from book


%=================== Header and Footer
\usepackage{fancyhdr}
\pagestyle{fancy}
\fancyhf{}
\lhead{ \textsf{\large\myname}\\
    Math 387\quad Analysis I \\
     Homework \hwnum }
\rhead{H/T: \collaborators\\
    \semester\\
    Due: \duedate}
\chead{ \underline{\textsf{Problem List}}\\[5pt] \problist}
\rfoot{Page \thepage}

%=================== Section/Subsection Formatting
\usepackage[explicit]{titlesec}

\titlespacing\section{0pt}{0 pt}{-30 pt}
\titleformat{\section}
  {}{}{0em}{}%#1 to diplay "Homework #"
%  {\centering \LARGE\bfseries}{}{1em}{}%#1 to diplay "Homework #"

\titleformat{\subsection}
  {\large\bfseries}{\thesubsection}{1em}{Problem {#1}}



%=================== Solution environment
\usepackage{mdframed}
\newenvironment{solution}
    {\begin{mdframed}\begin{proof}[\itshape Solution]}
    {\end{proof}\end{mdframed}}


\begin{document}
\setcounter{section}{\hwnum - 1}
\section{Homework \hwnum}
\emph{ \unassigned}


%--------------------------Question
\subsection{2.4.1}
Prove that $\{ \frac{n^2-1}{n^2} \}$ is Cauchy using directly the definition
of Cauchy sequences.

\iftoggle{showsolutions}{\begin{solution} 


\end{solution}}{}

%--------------------------Question
\subsection{2.4.2}
Let $\{ x_n \}$ be a sequence such that
there exists a $0 < C < 1$ such that
\begin{equation*}
\abs{x_{n+1} - x_n} \leq C \abs{x_{n}-x_{n-1}} .
\end{equation*}
Prove that $\{ x_n \}$ is Cauchy.
Hint:  You can freely use the formula (for $C \not= 1$)
\begin{equation*}
1+ C+ C^2 + \cdots + C^n = \frac{1-C^{n+1}}{1-C}.
\end{equation*}

\iftoggle{showsolutions}{\begin{solution} 


\end{solution}}{}

%--------------------------Question
\subsection{2.4.4}
Let $\{ x_n \}$ and $\{ y_n \}$ be sequences such
that $\lim\, y_n =0$.  Suppose that for all $k \in \N$
and
for all $m \geq k$ we have
\begin{equation*}
\abs{x_m-x_k} \leq y_k .
\end{equation*}
Show that $\{ x_n \}$ is Cauchy.

\iftoggle{showsolutions}{\begin{solution} 


\end{solution}}{}

%--------------------------Question
\subsection{2.5.1}
For $r \not= 1$, prove
\begin{equation*}
\sum_{k=0}^{n-1} r^k = \frac{1-r^n}{1-r} .
\end{equation*}
Hint:
Let $s := \sum_{k=0}^{n-1} r^k$, then
compute $s(1-r) = s-rs$, and solve for $s$.

\iftoggle{showsolutions}{\begin{solution} 


\end{solution}}{}

%--------------------------Question
\subsection{2.5.2}
Prove that for $-1 < r < 1$ we have
\begin{equation*}
\sum_{n=0}^\infty r^n = \frac{1}{1-r} .
\end{equation*}
Hint:  Use the previous exercise.

\iftoggle{showsolutions}{\begin{solution} 


\end{solution}}{}




%QUESTION TEMPLATE FOR FUTURE USE
%--------------------------Question
%\subsection{}
%\iftoggle{showsolutions}{\begin{solution} 
%
%\end{solution}}{}



\end{document}