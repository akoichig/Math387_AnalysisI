\documentclass{article}
\newcommand{\myname}{Your Name Goes Here}
\newcommand{\collaborators}{Last Names}


\newcommand{\semester}{Spring 2016}
\newcommand{\hwnum}{6}
\newcommand{\duedate}{Wednesday, March 9}
\newcommand{\problist}{
{\bf 3.1} \{2,3,5\}\\
{\bf 3.2} \{1,2,3,11\}
}

\newcommand{\unassigned}{
\flushleft Unassigned, but suggested: Problems 1,,7,8,10 in Section 3.1\\
Unassigned, but suggested: Problems 4,5,12 in Section 3.2
}
% The Course page: http://home.manhattan.edu/~andrew.greene/387/ also suggests some unassigned problems.

%=============================================================================
\usepackage{etoolbox}
\newtoggle{showsolutions}
%== Show solutions? Comment out one of the lines below

\toggletrue{showsolutions}
%\togglefalse{showsolutions}

%== The proofs appear below in between "\begin{solution}" and "\end{solution}."
%=============================================================================


\usepackage[utf8]{inputenc}
\usepackage{amsmath,amssymb,amsthm}
\usepackage{nicefrac}
\usepackage{calc}
\usepackage{enumerate}
\usepackage{graphicx}
\usepackage{verbatim}
\usepackage[draft]{hyperref}
\usepackage{color}
\usepackage[colorinlistoftodos]{todonotes}
\RequirePackage[paper=letterpaper, margin=.75in, headsep=.5in,voffset=0.5in]{geometry}


%=================== Useful Commands
% sets (some)
\newcommand{\C}{{\mathbb{C}}}
\newcommand{\R}{{\mathbb{R}}}
\newcommand{\Z}{{\mathbb{Z}}}
\newcommand{\N}{{\mathbb{N}}}
\newcommand{\Q}{{\mathbb{Q}}}
% reals
\newcommand{\sabs}[1]{\lvert {#1} \rvert}
\newcommand{\snorm}[1]{\lVert {#1} \rVert}
\newcommand{\abs}[1]{\left\lvert {#1} \right\rvert}
\newcommand{\norm}[1]{\left\lVert {#1} \right\rVert}
\newcommand{\myindex}[1]{#1} %ignores myindex command from book


%=================== Header and Footer
\usepackage{fancyhdr}
\pagestyle{fancy}
\fancyhf{}
\lhead{ \textsf{\large\myname}\\
    Math 387\quad Analysis I \\
     Homework \hwnum }
\rhead{H/T: \collaborators\\
    \semester\\
    Due: \duedate}
\chead{ \underline{\textsf{Problem List}}\\[5pt] \problist}
\rfoot{Page \thepage}

%=================== Section/Subsection Formatting
\usepackage[explicit]{titlesec}

\titlespacing\section{0pt}{0 pt}{-30 pt}
\titleformat{\section}
  {}{}{0em}{}%#1 to diplay "Homework #"
%  {\centering \LARGE\bfseries}{}{1em}{}%#1 to diplay "Homework #"

\titleformat{\subsection}
  {\large\bfseries}{\thesubsection}{1em}{Problem {#1}}



%=================== Solution environment
\usepackage{mdframed}
\newenvironment{solution}
    {\begin{mdframed}\begin{proof}[\itshape Solution]}
    {\end{proof}\end{mdframed}}


\begin{document}
\setcounter{section}{\hwnum - 1}
\section{Homework \hwnum}
\emph{ \unassigned}


%--------------------------Question
\subsection{3.1.2}
Prove Corollary 3.1.10. %\corref{fconstineq:cor}.

\iftoggle{showsolutions}{\begin{solution} 


\end{solution}}{}

%--------------------------Question
\subsection{3.1.3}
Prove Corollary 3.1.11. %\corref{fsqueeze:cor}.

\iftoggle{showsolutions}{\begin{solution} 


\end{solution}}{}

%--------------------------Question
\subsection{3.1.5}
Let $A \subset S$.  Show that if $c$ is a cluster point of $A$, then $c$
is a cluster point of $S$.  Note the difference from
Proposition 3.1.14.%\propref{prop:limrest}.

\iftoggle{showsolutions}{\begin{solution} 


\end{solution}}{}

%--------------------------Question
\subsection{3.2.1}
Using the definition of continuity directly prove that
$f \colon \R \to \R$ defined by
$f(x) := x^2$ is continuous.

\iftoggle{showsolutions}{\begin{solution} 


\end{solution}}{}

%--------------------------Question
\subsection{3.2.2}
Using the definition of continuity directly prove that
$f \colon (0,\infty) \to \R$ defined by
$f(x) := \nicefrac{1}{x}$ is continuous.

\iftoggle{showsolutions}{\begin{solution} 


\end{solution}}{}

%--------------------------Question
\subsection{3.2.3}
Let $f \colon \R \to \R$ be defined by
\begin{equation*}
f(x) :=
\begin{cases}
x & \text{ if $x$ is rational,} \\
x^2 & \text{ if $x$ is irrational.}
\end{cases}
\end{equation*}
Using the definition of continuity directly prove that
$f$ is continuous at $1$ and discontinuous at $2$.

\iftoggle{showsolutions}{\begin{solution} 


\end{solution}}{}

%--------------------------Question
\subsection{3.2.11}
Let $f \colon \R \to \R$ be continuous.  Suppose $f(c) > 0$.  Show that
there exists an $\alpha > 0$ such that for all $x \in (c-\alpha,c+\alpha)$
we have $f(x) > 0$.

\iftoggle{showsolutions}{\begin{solution} 


\end{solution}}{}




%QUESTION TEMPLATE FOR FUTURE USE
%--------------------------Question
%\subsection{}
%\iftoggle{showsolutions}{\begin{solution} 
%
%\end{solution}}{}



\end{document}