\documentclass{article}
\newcommand{\myname}{Your Name Goes Here}
\newcommand{\collaborators}{Last Names}


\newcommand{\semester}{Spring 2016}
\newcommand{\hwnum}{9}
\newcommand{\duedate}{Wednesday, April 6}
\newcommand{\problist}{
{\bf 5.1} \{1,2,7,10,11\}}

\newcommand{\unassigned}{
\flushleft 
Unassigned, but suggested: Problems 3,5 in Section 5.1
}
% The Course page: http://home.manhattan.edu/~andrew.greene/387/ also suggests some unassigned problems.

%=============================================================================
\usepackage{etoolbox}
\newtoggle{showsolutions}
%== Show solutions? Comment out one of the lines below

\toggletrue{showsolutions}
%\togglefalse{showsolutions}

%== The proofs appear below in between "\begin{solution}" and "\end{solution}."
%=============================================================================


\usepackage[utf8]{inputenc}
\usepackage{amsmath,amssymb,amsthm}
\usepackage{nicefrac}
\usepackage{calc}
\usepackage{enumerate}
\usepackage{graphicx}
\usepackage{verbatim}
\usepackage[draft]{hyperref}
\usepackage{color}
\usepackage[colorinlistoftodos]{todonotes}
\RequirePackage[paper=letterpaper, margin=.75in, headsep=.5in,voffset=0.5in]{geometry}


%=================== Useful Commands
% sets (some)
\newcommand{\C}{{\mathbb{C}}}
\newcommand{\R}{{\mathbb{R}}}
\newcommand{\Z}{{\mathbb{Z}}}
\newcommand{\N}{{\mathbb{N}}}
\newcommand{\Q}{{\mathbb{Q}}}
% reals
\newcommand{\sabs}[1]{\lvert {#1} \rvert}
\newcommand{\snorm}[1]{\lVert {#1} \rVert}
\newcommand{\abs}[1]{\left\lvert {#1} \right\rvert}
\newcommand{\norm}[1]{\left\lVert {#1} \right\rVert}
\newcommand{\myindex}[1]{#1} %ignores myindex command from book


%=================== Header and Footer
\usepackage{fancyhdr}
\pagestyle{fancy}
\fancyhf{}
\lhead{ \textsf{\large\myname}\\
    Math 387\quad Analysis I \\
     Homework \hwnum }
\rhead{H/T: \collaborators\\
    \semester\\
    Due: \duedate}
\chead{ \underline{\textsf{Problem List}}\\[5pt] \problist}
\rfoot{Page \thepage}

%=================== Section/Subsection Formatting
\usepackage[explicit]{titlesec}

\titlespacing\section{0pt}{0 pt}{-30 pt}
\titleformat{\section}
  {}{}{0em}{}%#1 to diplay "Homework #"
%  {\centering \LARGE\bfseries}{}{1em}{}%#1 to diplay "Homework #"

\titleformat{\subsection}
  {\large\bfseries}{\thesubsection}{1em}{Problem {#1}}



%=================== Solution environment
\usepackage{mdframed}
\newenvironment{solution}
    {\begin{mdframed}\begin{proof}[\itshape Solution]}
    {\end{proof}\end{mdframed}}


\begin{document}
\setcounter{section}{\hwnum - 1}
\section{Homework \hwnum}
\emph{ \unassigned}


%--------------------------Question
\subsection{5.1.1}

Let $f \colon [0,1] \to \R$ be defined by $f(x) := x^3$
and let $P := \{ 0, 0.1, 0.4, 1 \}$.  Compute $L(P,f)$ and $U(P,f)$.


\iftoggle{showsolutions}{\begin{solution}


\end{solution}}{}

%--------------------------Question
\subsection{5.1.2}

Let $f \colon [0,1] \to \R$ be defined by $f(x) := x$.
Show that $f \in \mathcal R[0,1]$ and
compute $\int_0^1 f$ using the definition of the integral
(but
feel free to use the propositions of this section).%\propref{intulbound:prop}).


\iftoggle{showsolutions}{\begin{solution}


\end{solution}}{}


%--------------------------Question
\subsection{5.1.7}

Suppose $f \colon [a,b] \to \R$ is Riemann integrable.  Let $\epsilon
> 0$ be given.  Then show that there exists a partition $P = \{ x_0, x_1,
\ldots, x_n \}$
such that if we
pick any set of numbers $\{ c_1, c_2, \ldots, c_n \}$ with
$c_k \in [x_{k-1},x_k]$ for all $k$, then
\begin{equation*}
\abs{\int_a^b f - \sum_{k=1}^n f(c_k) \Delta x_k} < \epsilon .
\end{equation*}


\iftoggle{showsolutions}{\begin{solution}


\end{solution}}{}

%--------------------------Question
\subsection{5.1.10}

Let $f \colon [0,1] \to \R$ be a bounded function.
Let $P_n = \{ x_0,x_1,\ldots,x_n \}$ be a uniform partition of $[0,1]$,
that is, $x_j := \nicefrac{j}{n}$.  Is $\{ L(P_n,f) \}_{n=1}^\infty$
always monotone?  Yes/No: Prove or find a counterexample.


\iftoggle{showsolutions}{\begin{solution}


\end{solution}}{}

%--------------------------Question
\subsection{5.1.11}

For a bounded function $f \colon [0,1] \to \R$ let
$R_n := (\nicefrac{1}{n})\sum_{j=1}^n f(\nicefrac{j}{n})$ (the
uniform right hand rule).
a) If $f$ is Riemann integrable show $\int_0^1 f = \lim \, R_n$.
b) Find an $f$ that is not Riemann integrable, but $\lim \, R_n$ exists.


\iftoggle{showsolutions}{\begin{solution}


\end{solution}}{}



%QUESTION TEMPLATE FOR FUTURE USE
%--------------------------Question
%\subsection{}
%\iftoggle{showsolutions}{\begin{solution} 
%
%\end{solution}}{}



\end{document}