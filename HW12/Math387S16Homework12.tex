\documentclass{article}
\newcommand{\myname}{Your Name Goes Here}
\newcommand{\collaborators}{Last Names}


\newcommand{\semester}{Spring 2016}
\newcommand{\hwnum}{12}
\newcommand{\duedate}{Wednesday, April 27}
\newcommand{\problist}{
{\bf 6.1} \{1,2,7\}
{\bf 6.2} \{1,2\}
{\bf 7.1} \{1,3,7\}}

\newcommand{\unassigned}{
\flushleft 
{\color{red}Note: You can turn in the 7.1 Problems this week or with the next homework.}\\
Unassigned, but suggested: Problems 4,5,10,11 in Section 6.1\\
Unassigned, but suggested: Problems 3,5 in Section 6.2\\
Unassigned, but suggested: Problems 6,7,8 in Section 7.1
}
% The Course page: http://home.manhattan.edu/~andrew.greene/387/ also suggests some unassigned problems.

%=============================================================================
\usepackage{etoolbox}
\newtoggle{showsolutions}
%== Show solutions? Comment out one of the lines below

\toggletrue{showsolutions}
%\togglefalse{showsolutions}

%== The proofs appear below in between "\begin{solution}" and "\end{solution}."
%=============================================================================


\usepackage[utf8]{inputenc}
\usepackage{amsmath,amssymb,amsthm}
\usepackage{nicefrac}
\usepackage{calc}
\usepackage{enumerate}
\usepackage{graphicx}
\usepackage{verbatim}
\usepackage[draft]{hyperref}
\usepackage{color}
\usepackage[colorinlistoftodos]{todonotes}
\RequirePackage[paper=letterpaper, margin=.75in, headsep=.5in,voffset=0.5in]{geometry}


%=================== Useful Commands
% sets (some)
\newcommand{\C}{{\mathbb{C}}}
\newcommand{\R}{{\mathbb{R}}}
\newcommand{\Z}{{\mathbb{Z}}}
\newcommand{\N}{{\mathbb{N}}}
\newcommand{\Q}{{\mathbb{Q}}}
% reals
\newcommand{\sabs}[1]{\lvert {#1} \rvert}
\newcommand{\snorm}[1]{\lVert {#1} \rVert}
\newcommand{\abs}[1]{\left\lvert {#1} \right\rvert}
\newcommand{\norm}[1]{\left\lVert {#1} \right\rVert}
\newcommand{\myindex}[1]{#1} %ignores myindex command from book
\newcommand{\sR}{{\mathcal{R}}} %Riemann Integrable functions
\newcommand{\rednote}[1]{{\color{red} #1 }}

%=================== Header and Footer
\usepackage{fancyhdr}
\pagestyle{fancy}
\fancyhf{}
\lhead{ \textsf{\large\myname}\\
    Math 387\quad Analysis I \\
     Homework \hwnum }
\rhead{H/T: \collaborators\\
    \semester\\
    Due: \duedate}
\chead{ \underline{\textsf{Problem List}}\\[5pt] \problist}
\rfoot{Page \thepage}

%=================== Section/Subsection Formatting
\usepackage[explicit]{titlesec}

\titlespacing\section{0pt}{0 pt}{-30 pt}
\titleformat{\section}
  {}{}{0em}{}%#1 to diplay "Homework #"
%  {\centering \LARGE\bfseries}{}{1em}{}%#1 to diplay "Homework #"

\titleformat{\subsection}
  {\large\bfseries}{\thesubsection}{1em}{Problem {#1}}



%=================== Solution environment
\usepackage{mdframed}
\newenvironment{solution}
    {\begin{mdframed}\begin{proof}[\itshape Solution]}
    {\end{proof}\end{mdframed}}


\begin{document}
\setcounter{section}{11}
\section{Homework 12}
\emph{ \unassigned}

%--------------------------Question
\subsection{6.1.1}

Let $f$ and $g$ be bounded functions on $[a,b]$.  Prove 
\begin{equation*}
\norm{f+g}_u \leq \norm{f}_u + \norm{g}_u .
\end{equation*}


\iftoggle{showsolutions}{\begin{solution}


\end{solution}}{}

%--------------------------Question
\subsection{6.1.2}

a) Find the pointwise limit $\dfrac{e^{x/n}}{n}$ for $x \in \R$. \nopagebreak
\\
b) Is the limit uniform on $\R$?
\\
c) Is the limit uniform on $[0,1]$?


\iftoggle{showsolutions}{\begin{solution}


\end{solution}}{}

%--------------------------Question
\subsection{6.1.7}

Suppose there exists a sequence of functions $\{ g_n \}$ uniformly
converging to $0$ on $A$.  Now suppose we have a sequence of functions
$\{ f_n \}$ and a function $f$ on $A$ such that
\begin{equation*}
\abs{f_n(x) - f(x)} \leq g_n(x) 
\end{equation*}
for all $x \in A$.  Show that $\{ f_n \}$ converges uniformly to $f$ on $A$.


\iftoggle{showsolutions}{\begin{solution}


\end{solution}}{}

%--------------------------Question
\subsection{6.2.1}

While uniform convergence preserves continuity, it does not preserve
differentiability.  Find an explicit example of a sequence of
differentiable functions on $[-1,1]$ that converge uniformly to
a function $f$ such that $f$ is not differentiable.  Hint: Consider
$\abs{x}^{1+1/n}$, show that these functions are differentiable,
converge uniformly, and then show that the limit is not differentiable.


\iftoggle{showsolutions}{\begin{solution}


\end{solution}}{}

%--------------------------Question
\subsection{6.2.2}

Let $f_n(x) = \frac{x^n}{n}$.  Show that $\{ f_n \}$ converges uniformly to
a differentiable function $f$ on $[0,1]$ (find $f$).  However, show that
$\displaystyle f'(1) \not= \lim_{n\to\infty} f_n'(1)$.


\iftoggle{showsolutions}{\begin{solution}


\end{solution}}{}

%--------------------------Question
\subsection{7.1.1}

Show that for any set $X$, the discrete metric ($d(x,y) = 1$ if $x\not=y$ and
$d(x,x) = 0$) does give a metric space $(X,d)$.


\iftoggle{showsolutions}{\begin{solution}


\end{solution}}{}

%--------------------------Question
\subsection{7.1.3}

Let $X := \{ a, b \}$ be a set.  Can you make it into two distinct metric
spaces?  (define two distinct metrics on it)


\iftoggle{showsolutions}{\begin{solution}


\end{solution}}{}

%--------------------------Question
\subsection{7.1.7}

Let $X$ be the set of continuous functions on $[0,1]$.  Let $\varphi \colon
[0,1] \to (0,\infty)$ be continuous.  Define
\begin{equation*}
d(f,g) := \int_0^1 \abs{f(x)-g(x)}\varphi(x)~dx .
\end{equation*}
Show that $(X,d)$ is a metric space.


\iftoggle{showsolutions}{\begin{solution}


\end{solution}}{}





%QUESTION TEMPLATE FOR FUTURE USE
%--------------------------Question
%\subsection{}
%\iftoggle{showsolutions}{\begin{solution} 
%
%\end{solution}}{}



\end{document}